\documentclass[12pt]{article}

\usepackage[top=1in, bottom=1in, left=1in, right=1in]{geometry}
\usepackage{amsfonts,amsmath,amssymb}
\usepackage[none]{hyphenat} 
\usepackage[nottoc,notlot,notlof]{tocbibind}
\usepackage{graphicx}
\usepackage{xcolor}
\usepackage{yhmath}
\usepackage{mathdots}
\usepackage{MnSymbol}
\usepackage{etoolbox}
\usepackage{esint}
\usepackage{cleveref}
\usepackage{pgf}
\usepackage{verbatim}
\usepackage{environ}
\usepackage{tcolorbox}
\usepackage{mathtools}




\numberwithin{equation}{subsection}

\newcommand{\lap}{\mathscr{L}}
\newcommand{\lnp}[1]{\ln\left( #1 \right)}
\newcommand{\sinp}[1]{\sin\left( #1 \right)}
\newcommand{\cosp}[1]{\cos\left( #1 \right)}
\newcommand{\tanp}[1]{\tan\left( #1 \right)}
\newcommand{\secp}[1]{\sec\left( #1 \right)}
\newcommand{\cscp}[1]{\csc\left( #1 \right)}
\newcommand{\cotp}[1]{\cot\left( #1 \right)}
\newcommand{\vecp}[1]{\langle #1 \rangle}
\newcommand{\magp}[1]{\| #1 \|}
\newcommand{\absp}[1]{\left\vert #1 \right\vert}
\newcommand{\parx}[1]{\frac{\partial #1}{\partial x}}
\newcommand{\pary}[1]{\frac{\partial #1}{\partial y}}
\newcommand{\parz}[1]{\frac{\partial #1}{\partial z}}
\newcommand{\derx}[1]{\frac{d #1}{dx}}
\newcommand{\dery}[1]{\frac{d #1}{dy}}
\newcommand{\dert}[1]{\frac{d #1}{dt}}
\newcommand{\deryx}{\frac{dy}{dx}}
\newcommand{\inda}{\hspace{.5cm}}
\newcommand{\indb}{\hspace{1cm}}
\newcommand{\indc}{\hspace{1.5cm}}
\newcommand{\indd}{\hspace{2cm}}
\newcommand{\inde}{\hspace{2.5cm}}
\newcommand{\indf}{\hspace{3cm}}
\newcommand{\indg}{\hspace{3.5cm}}
\newcommand{\indh}{\hspace{4cm}}
\newcommand{\indi}{\hspace{4.5cm}}
\newcommand{\indj}{\hspace{5cm}}
\newcommand{\indk}{\hspace{5.5cm}}
\newcommand{\indl}{\hspace{6cm}}
\newcommand{\indm}{\hspace{6.5cm}}
\newcommand{\indn}{\hspace{7cm}}
\newcommand{\indo}{\hspace{7.5cm}}
\newcommand{\indp}{\hspace{8cm}}
\newcommand{\indq}{\hspace{8.5cm}}
\newcommand{\indr}{\hspace{9cm}}
\newcommand{\inds}{\hspace{9.5cm}}
\newcommand{\indt}{\hspace{10cm}}
\newcommand{\indu}{\hspace{10.5cm}}
\newcommand{\indv}{\hspace{11cm}}
\newcommand{\indw}{\hspace{11.5cm}}
\newcommand{\indx}{\hspace{12cm}}
\newcommand{\indy}{\hspace{12.5cm}}
\newcommand{\indz}{\hspace{13cm}}
\newcommand{\exa}{\noindent \underline{Example}: \hspace{1cm}}
\newcommand{\longsquiggly}{\xymatrix{{}\ar@{~>}[r]&{}}}



\DeclareMathOperator{\arcsec}{arcsec}
\DeclareMathOperator{\arccot}{arccot}
\DeclareMathOperator{\arccsc}{arccsc}
\DeclareMathOperator{\sech}{sech}
\DeclareMathOperator{\csch}{csch}
\DeclareMathOperator{\arcsinh}{arcsinh}
\DeclareMathOperator{\arccosh}{arccosh}
\DeclareMathOperator{\arctanh}{arctanh}
\DeclareMathOperator{\arccsch}{arccsch}
\DeclareMathOperator{\arcsech}{arcsech}
\DeclareMathOperator{\arccoth}{arccoth}
\DeclareMathOperator{\expo}{exp}
\DeclareMathOperator{\trace}{trace}
\DeclareMathOperator{\vdiv}{div}
\DeclareMathOperator{\vcurl}{curl}


\newcommand{\expop}[1]{\expo \left( #1 \right)}




\begin{document}

\begin{titlepage}
\begin{center}
\begin{Large}
Mathematics\\
\end{Large}
\vfill
\line(1,0){469.75502}\\[1mm]
\begin{Large}
\textbf{Math Review Packet}
\end{Large}
\line(1,0){469.75502}
\vfill
By Michael Kasprzak,\\
February 18, 2019
\end{center}
\end{titlepage}


\pagenumbering{roman}


\tableofcontents
\cleardoublepage


\newpage
\pagenumbering{arabic}
\setcounter{page}{1}
\section{Trigonometry}
\subsection{Trig Identities and Formulas}




\begin{flushleft}
Cofunction Identities
\begin{equation}
\sinp{\theta}=\cosp{\frac{\pi}{2}-\theta}
\end{equation}
\begin{equation}
\cosp{\theta}=\sinp{\frac{\pi}{2}-\theta}
\end{equation}
\begin{equation}
\tanp{\theta}=\cotp{\frac{\pi}{2}-\theta}
\end{equation}
\begin{equation}
\cscp{\theta}=\secp{\frac{\pi}{2}-\theta}
\end{equation}
\begin{equation}
\secp{\theta}=\cscp{\frac{\pi}{2}-\theta}
\end{equation}
\begin{equation}
\cotp{\theta}=\tanp{\frac{\pi}{2}-\theta}
\end{equation}

Reciprocal Identities
\begin{equation}
\sinp{\theta}=\frac{1}{\cscp{\theta}}
\end{equation}
\begin{equation}
\cosp{\theta}=\frac{1}{\secp{\theta}}
\end{equation}
\begin{equation}
\tanp{\theta}=\frac{1}{\cotp{\theta}}
\end{equation}
\begin{equation}
\cscp{\theta}=\frac{1}{\sinp{\theta}}
\end{equation}
\begin{equation}
\secp{\theta}=\frac{1}{\cosp{\theta}}
\end{equation}
\begin{equation}
\cotp{\theta}=\frac{1}{\tanp{\theta}}
\end{equation}

Quotient Identities
\begin{equation}
\tanp{\theta}=\frac{\sinp{\theta}}{\cosp{\theta}}
\end{equation}
\begin{equation}
\cotp{\theta}=\frac{\cosp{\theta}}{\sinp{\theta}}
\end{equation}

Pythagorean Identities
\begin{equation}
\sin^{2}{\theta}+\cos^2{\theta}=1
\end{equation}
\begin{equation}
1+\tan^2{\theta}=\sec^2{\theta}
\end{equation}
\begin{equation}
1+\cot^2{\theta}=\csc^2{\theta}
\end{equation}
\end{flushleft}

\newpage
Sum and Difference Formulas
\begin{equation}
\sinp{A \pm B}= \sin{A} \cos{B} \pm \cos{A} \sin{B}
\end{equation}
\begin{equation}
\cosp{A \pm B}= \cos{A} \cos{B} \mp \sin{A} \sin{B}
\end{equation}
\begin{equation}
\tanp{A \pm B}=\frac{\tan{A} \pm \tan{B}}{1 \mp \tan{A} \tan{B}}
\end{equation}

Odd/Even Formulas
\begin{equation}
\sinp{-\theta}=-\sinp{\theta}
\end{equation}
\begin{equation}
\cosp{-\theta}=\cosp{\theta}
\end{equation}
\begin{equation}
\tanp{-\theta}=-\tanp{\theta}
\end{equation}
\begin{equation}
\cscp{-\theta}=-\cscp{\theta}
\end{equation}
\begin{equation}
\secp{-\theta}=\secp{\theta}
\end{equation}
\begin{equation}
\cotp{-\theta}=-\cotp{\theta}
\end{equation}

Double Angle Formula
\begin{equation}
\sinp{2\theta}=2\sin{\theta} \cos{\theta}
\end{equation}
\begin{equation}
\cosp{2\theta}=\cos^2{\theta}-\sin^2{\theta}
\end{equation}
\begin{equation}
\ \ \ \ \ \ \ \ = 2\cos^2\theta - 1
\end{equation}
\begin{equation}
\ \ \ \ \ \ \ \ = 1 - 2\sin^2 \theta
\end{equation}
\begin{equation}
\tanp{2\theta}=\frac{2\tan{\theta}}{1-\tan^2{\theta}}
\end{equation}

Half-Angle Formulas
\begin{equation}
\sinp{\frac{\theta}{2}}= \pm \sqrt{\frac{1-\cos{\theta}}{2}}
\end{equation}
\begin{equation}
\cosp{\frac{\theta}{2}}= \pm \sqrt{\frac{1+\cos{\theta}}{2}}
\end{equation}
\begin{equation}
\tanp{\frac{\theta}{2}}= \frac{1-\cos{\theta}}{\sin{\theta}}=\frac{\sin{\theta}}{1+\cos{\theta}}
\end{equation}

Squared Identities
\begin{equation}
\sin^2{\theta}=\frac{1-\cos{2\theta}}{2}
\end{equation}
\begin{equation}
\cos^2{\theta}=\frac{1+\cos{2\theta}}{2}
\end{equation}
\begin{equation}
\tan^2{\theta}=\frac{1-\cos{2\theta}}{1+\cos{2\theta}}
\end{equation}
\newpage

Product to Sum Formulas
\begin{equation}
\sin{A} \sin{B} = \frac{1}{2}[\cosp{A-B}-\cosp{A+B}]
\end{equation}
\begin{equation}
\cos{A} \cos{B} = \frac{1}{2}[\cosp{A-B}+\cosp{A+B}]
\end{equation}
\begin{equation}
\sin{A} \cos{B} = \frac{1}{2}[\sinp{A+B}+\sinp{A-B}]
\end{equation}
\begin{equation}
\cos{A} \sin{B} = \frac{1}{2}[\sinp{A+B}-\sinp{A-B}]
\end{equation}

Sum to Product Identities
\begin{equation}
\sin{A} + \sin{B} = 2\sinp{\frac{A+B}{2}} \cosp{\frac{A-B}{2}}
\end{equation}
\begin{equation}
\sin{A} - \sin{B} = 2\cosp{\frac{A+B}{2}} \sinp{\frac{A-B}{2}}
\end{equation}
\begin{equation}
\cos{A} + \cos{B} = 2\cosp{\frac{A+B}{2}} \cosp{\frac{A-B}{2}}
\end{equation}
\begin{equation}
\cos{A} + \cos{B} = 2\sinp{\frac{A+B}{2}} \sinp{\frac{A-B}{2}}
\end{equation}

\subsection{Hyperbolic Trigonometry}

Functions and Definitions
\begin{equation}
\sinh{x}=\frac{e^{x}-e^{-x}}{2}
\end{equation}
\begin{equation}
\cosh{x}=\frac{e^{x}+e^{-x}}{2}
\end{equation}
\begin{equation}
\tanh{x}=\frac{e^{x}-e^{-x}}{e^{x}+e^{-x}}
\end{equation}
\begin{equation}
\csch{x}=\frac{2}{e^{x}-e^{-x}}
\end{equation}
\begin{equation}
\sech{x}=\frac{2}{e^{x}+e^{-x}}
\end{equation}
\begin{equation}
\coth{x}=\frac{e^{x}+e^{-x}}{e^{x}-e^{-x}}
\end{equation}
\newpage


Definitions of Inverse Functions
\begin{equation}
\arcsinh{x}=\lnp{x+ \sqrt{x^{2}+1}}
\end{equation}
\begin{equation}
\arccosh{x}=\lnp{x \pm \sqrt{x^{2}-1}}
\end{equation}
\begin{equation}
\arctanh{x}= \frac{1}{2} \lnp{\frac{1+x}{1-x}}
\end{equation}
\begin{equation}
\arccsch{x}=\lnp{\frac{1+\sqrt{x^{2}+1}}{x}}
\end{equation}
\begin{equation}
\arcsech{x}=\lnp{\frac{1 \pm \sqrt{x^{2}+1}}{x}}
\end{equation}
\begin{equation}
\arccoth{x}= \frac{1}{2} \lnp{\frac{x+1}{x-1}}
\end{equation}

Hyperbolic Identities
\begin{equation}
\sinh{-x}=-\sinh{x}
\end{equation}
\begin{equation}
\cosh{-x}=\cosh{x}
\end{equation}
\begin{equation}
\cosh^2{x}-\sinh^2{x}=1
\end{equation}
\begin{equation}
\sech^2{x}=1-\tanh^2{x}
\end{equation}
\begin{equation}
\sinh{x+y}=\sinh{x}\cosh{y} + \cosh{x}\sinh{y}
\end{equation}
\begin{equation}
\cosh{x+y}=\cosh{x}\cosh{y} + \sinh{x}\sinh{y}
\end{equation}
\begin{equation}
\sinh{2x}=2\sinh{x}\cosh{x}
\end{equation}
\begin{equation}
\cosh{2x}=\sinh^2{x}+\cosh^2{x}
\end{equation}
\begin{equation}
\sinh^2{x}=\frac{1}{2}(-1+\cosh{2x})
\end{equation}
\begin{equation}
\cosh^2{x}=\frac{1}{2}(1+\cosh{2x})
\end{equation}

\newpage



\section{Matrices}
\subsection{Basic Operations}
\begin{flushleft}
Adding and Subtracting Matrices
\begin{equation}
\begin{bmatrix}
a & b \\
c & d \\
\end{bmatrix} 
+
\begin{bmatrix}
e & f \\
g & h \\
\end{bmatrix}
=
\begin{bmatrix}
a+e & b+f \\
c+g & d+h \\
\end{bmatrix}
\end{equation}

\begin{equation}
\begin{bmatrix}
a & b \\
c & d \\
\end{bmatrix} 
-
\begin{bmatrix}
e & f \\
g & h \\
\end{bmatrix}
=
\begin{bmatrix}
a-e & b-f \\
c-g & d-h \\
\end{bmatrix}
\end{equation}

Multiplying Matrices
\begin{equation}
\begin{bmatrix}
a & b \\
c & d \\
\end{bmatrix} 
\cdot
\begin{bmatrix}
e & f \\
g & h \\
\end{bmatrix}
=
\begin{bmatrix}
ae+bg & af+bh \\
ce+dg & cf+dh \\
\end{bmatrix}
\end{equation}

Properties of Matrix Operations\\
 \bigskip 
Associative Property of Addition: \ \ \  \ \ \ \ \ \ \ \ \ \ \ \ \ \ \ \ \ \ \ \ $(A+B)+C=A+(B+C)$\\
Commutative Property of Addition: \  \ \ \ \ \ \ \  \ \ \ \ \ \ \ \ \ \ \ \ $A+B=B+A$\\
Distributive Property: \ \ \ \ \ \ \ \ \ \ \ \ \ \ \ \ \ \ \ \ \ \ \ \ \ \ \ \ \ \ \ \ \ \ \ \ \ \ $k(A \pm B)= kA \pm kB$\\
Associative Property of Matrix Multiplication:\ \ \ \ \ \ \   $A(BC)=(AB)C$\\
Left Distribution Property: \ \ \ \ \ \ \ \ \ \ \ \ \ \ \ \ \ \ \ \ \ \ \ \ \ \ \ \ \ \ \ $A(B+C)=AB+AC$\\
Right Distribution Property: \ \ \ \ \ \ \ \ \ \ \ \  \ \ \ \ \ \ \  \ \ \ \ \ \ \ \ \ \ $(A+B)C=AC+BC$\\
Associative Property of Scalar Multiplication: \ \ \ \ \ \ \ $k(AB)=(kA)B=A(kB)$\\
\bigskip

Determinants
\begin{equation}
\textrm{det} \begin{bmatrix}
a & b \\
c & d \\
\end{bmatrix}
=
\begin{bmatrix}
\textcolor{blue}{a} & \textcolor{red}{b}\\
\textcolor{red}{c} & \textcolor{blue}{d}\\
\end{bmatrix}
=
\textcolor{blue}{ad}-\textcolor{red}{bc}
\end{equation}

\begin{equation}
\textrm{det} \begin{bmatrix}
a & b & c \\
d & e & f \\
g & h & i \\
\end{bmatrix}
=
\begin{bmatrix}
\textcolor{blue}{\ddots} & \textcolor{blue}{\ddots} & \textcolor{blue}{\ddots} \\
\textcolor{blue}{\ddots} & \textcolor{blue}{\ddots} & \textcolor{blue}{\ddots} \\
\textcolor{blue}{\ddots} & \textcolor{blue}{\ddots} & \textcolor{blue}{\ddots} \\
\end{bmatrix}
=
\begin{bmatrix}
\textcolor{red}{\udots} & \textcolor{red}{\udots} & \textcolor{red}{\udots} \\
\textcolor{red}{\udots} & \textcolor{red}{\udots} & \textcolor{red}{\udots} \\
\textcolor{red}{\udots} & \textcolor{red}{\udots} & \textcolor{red}{\udots} \\
\end{bmatrix}=
\textcolor{blue}{(aei+bfg+cdh)}-\textcolor{red}{(gec+dbi+ahf)}
\end{equation}


\subsection{Applications of Matrices}


Area of Triangle with Vertices $(x_{1},y_{1}), \ (x_{2},y_{2}), \ (x_{3},y_{3})$
\begin{equation}
\textrm{Area}= \pm \frac{1}{2}
\begin{bmatrix}
x_{1} & y_{1} & 1 \\
x_{2} & y_{2} & 1 \\
x_{3} & y_{3} & 1\\
\end{bmatrix}
\end{equation} 
\end{flushleft}
\begin{center}
$\pm$ to get positive area
\end{center}
\begin{flushleft}


Cramer's Rule for 2x2 System
$$ax+by=e$$
$$cx+dy=f$$

$$\textrm{Coefficient Matrix: A} 
\begin{bmatrix}
a & b \\
c & d \\
\end{bmatrix}$$
\newpage

Solutions 
\begin{equation}
x=\frac{\begin{bmatrix}
e & b \\
f & d \\
\end{bmatrix}}{\textrm{detA}}
\end{equation}

\begin{equation}
y=\frac{\begin{bmatrix}
a & e \\
c & f \\
\end{bmatrix}}{\textrm{detA}}
\end{equation}

Cramer's Rule for 3x3 System
$$ax+by+cz=j$$
$$dx+ey+fz=k$$
$$gx+hy+iz=l$$

$$\textrm{Coefficient Matrix: A} 
\begin{bmatrix}
a & b & c\\
d & e & f \\
g & h & i \\
\end{bmatrix}$$

Solutions 
\begin{equation}
x=\frac{\begin{bmatrix}
j & b & c\\
k & e & f \\
l & h & i \\
\end{bmatrix}}{\textrm{detA}}
\end{equation}

\begin{equation}
y=\frac{\begin{bmatrix}
a & j & c\\
d & k & f \\
g & l & i \\
\end{bmatrix}}{\textrm{detA}}
\end{equation}

\begin{equation}
z=\frac{\begin{bmatrix}
a & b & j \\
d & e & k \\
g & h & l \\
\end{bmatrix}}{\textrm{detA}}
\end{equation}  

\subsection{Identity and Inverse Matrices}

Identity Matrix: AI=A IA=A
$$\textrm{2x2 Identity} \begin{bmatrix}
1 & 0 \\
0 & 1 \\
\end{bmatrix}$$
$$\textrm{3x3 Identity} \begin{bmatrix}
1 & 0 & 0 \\
0 & 1 & 0 \\
0 & 0 & 1 \\
\end{bmatrix}$$
Inverse Matrices: A and B are inverse matrices if AB=I and BA=I. A matrix only has an inverse is its dterminant $\neq 0$.
\newpage

Inverse of a 2x2 Matrix
$$A=\begin{bmatrix}
a & b \\
c & d \\
\end{bmatrix}$$

\begin{equation}
A^{-1}=\frac{1}{\vert A \vert }\begin{bmatrix}
d & -b \\
-c & a \\
\end{bmatrix}
=
\frac{1}{ad-cb }\begin{bmatrix}
d & -b \\
-c & a \\
\end{bmatrix}
\end{equation}

Using Inverse Matrices to Solve Systems of Equations
$$ax+by=e$$
$$cx+dy=f$$

$$\textrm{Coefficient Matrix:} \ A= 
\begin{bmatrix}
a & b \\
c & d \\
\end{bmatrix}$$
$$\textrm{Variable Matrix:} \ X=\begin{bmatrix}
x\\
y\\
\end{bmatrix}$$
$$\textrm{Matrix of Constants:} \ B=\begin{bmatrix}
e\\
f\\
\end{bmatrix}$$

1. Setup Equation 
\begin{equation}
AX=B
\end{equation}
2. Find $A^{-1}$\\
3. Multiply Each Side by $A^{-1}$
\begin{equation}
A^{-1}AX=A^{-1}B
\end{equation}
\begin{equation}
X=A^{-1}B
\end{equation}
\begin{equation}
X=\begin{bmatrix}
\frac{d}{\textrm{det A}} & \frac{-c}{\textrm{det A}} \\
\frac{-b}{\textrm{det A}} & \frac{a}{\textrm{det A}} \\
\end{bmatrix}
\begin{bmatrix}
e\\
f\\
\end{bmatrix}
=
\begin{bmatrix}
\frac{de-cf}{\textrm{detA}}\\
\frac{af-be}{\textrm{detA}}\\
\end{bmatrix}
= \begin{bmatrix}
x\\
y\\
\end{bmatrix}
\end{equation}

\begin{equation}
x=\frac{de-cf}{\textrm{detA}}
\end{equation}
\begin{equation}
y=\frac{af-be}{\textrm{detA}}
\end{equation}




\newpage
\end{flushleft}



\section{Derivatives}
\subsection{Basic Rules}
\begin{flushleft}
Constant Term
\begin{equation}
\frac{d}{dx}[C]=0
\end{equation}\\

Constant Multiple Rule
\begin{equation}
\frac{d}{dx}[Cf(x)]=C \cdot \frac{d}{dx}[f(x)]
\end{equation}

Sum/Difference Rule
\begin{equation}
\frac{d}{dx}[f(x)\pm g(x)]=\frac{d}{dx}[f(x)] \pm \frac{d}{dx}[g(x)]
\end{equation}

Power Rule
\begin{equation}
\frac{d}{dx}[ax^{n}]=n\cdot ax^{n-1}
\end{equation}

Product Rule
\begin{equation}
\frac{d}{dx}[f(x) \cdot g(x)]= f'(x)g(x)+f(x)g'(x)
\end{equation}

Quotient Rule
\begin{equation}
\frac{d}{dx}\biggl[ \frac{f(x)}{g(x)} \biggr]=\frac{f'(x)g(x)-f(x)g'(x)}{[g(x)]^{2}}
\end{equation}

Chain Rule
\begin{equation}
\frac{d}{dx}[f(g(x))]=f'(g(x)) \cdot g'(x)
\end{equation}

Exponentials
\begin{equation}
\frac{d}{dx}[a^{x}]=a^{x}\cdot \lnp{a}
\end{equation}

Logarithms
\begin{equation}
\frac{d}{dx}[\log_{a}{x}]=\frac{1}{x \cdot \lnp{a}}
\end{equation}

Generalized Power Rule
\begin{equation}
\frac{d}{dx}[f(x)^{g(x)}]=f(x)^{g(x)}\biggl[g'(x)\cdot\ln{f(x)}+\frac{g(x)f'(x)}{f(x)}\biggr]
\end{equation}

Inverse Functions
\begin{equation}
\frac{d}{dx}[f^{-1}(x)]= \frac{1}{f'(f^{-1}(x))}
\end{equation}

\newpage


\subsection{Trigonometric Functions}
Normal Trig Functions
\begin{equation}
\frac{d}{dx}[\sinp{x}]=\cosp{x}
\end{equation}
\begin{equation}
\frac{d}{dx}[\cosp{x}]=-\sinp{x}
\end{equation}
\begin{equation}
\frac{d}{dx}[\tanp{x}]=\sec^{2}\left( x \right)
\end{equation}
\begin{equation}
\frac{d}{dx}[\cscp{x} ]= -\cscp{x} \cotp{x}
\end{equation}
\begin{equation}
\frac{d}{dx}[\secp{x} ]= \secp{x} \tanp{x}
\end{equation}
\begin{equation}
\frac{d}{dx}[\cotp{x} ]= -\csc^{2}\left( x \right)
\end{equation}\\
\bigskip 


Inverse Trig Functions
\begin{equation}
\frac{d}{dx}[\arcsin{x}]=\frac{1}{\sqrt{1-x^{2}}}
\end{equation}
\begin{equation}
\frac{d}{dx}[\arccos{x}]=\frac{-1}{\sqrt{1-x^{2}}}
\end{equation}
\begin{equation}
\frac{d}{dx}[\arctan{x}]=\frac{1}{1+x^{2}}
\end{equation}
\begin{equation}
\frac{d}{dx}[\arccsc{x}]=\frac{-1}{ \mid x \mid \sqrt{x^{2}-1}}
\end{equation}
\begin{equation}
\frac{d}{dx}[\arcsec{x}]=\frac{1}{ \mid x \mid \sqrt{x^{2}-1}}
\end{equation}
\begin{equation}
\frac{d}{dx}[\arccot{x}]=\frac{-1}{1+x^{2}}
\end{equation}

\newpage

\subsection{Hyperbolic Functions}

Normal Hyperbolic Functions
\begin{equation}
\frac{d}{dx}[\sinh{x}]=\cosh{x}
\end{equation}
\begin{equation}
\frac{d}{dx}[\cosh{x}]=\sinh{x}
\end{equation}
\begin{equation}
\frac{d}{dx}[\tanh{x}]=\sech^{2}x
\end{equation}
\begin{equation}
\frac{d}{dx}[\csch{x}]=-\csch{x} \coth{x}
\end{equation}
\begin{equation}
\frac{d}{dx}[\sech{x}]=-\sech{x} \tanh{x}
\end{equation}
\begin{equation}
\frac{d}{dx}[\coth{x}]=-\csch^{2}x
\end{equation}\\
\bigskip 

Inverse Hyperbolic Functions
\begin{equation}
\frac{d}{dx}[\arcsinh{x}]=\frac{1}{\sqrt{1+x^{2}}}
\end{equation}
\begin{equation}
\frac{d}{dx}[\arccosh{x}]=\frac{1}{\sqrt{1-x^{2}}}
\end{equation}
\begin{equation}
\frac{d}{dx}[\arctanh{x}]=\frac{1}{1-x^{2}}
\end{equation}
\begin{equation}
\frac{d}{dx}[\arccsch{x}]=\frac{-1}{ x  \sqrt{1-x^{2}}}
\end{equation}
\begin{equation}
\frac{d}{dx}[\arcsech{x}]=\frac{-1}{ x \sqrt{1-x^{2}}}
\end{equation}
\begin{equation}
\frac{d}{dx}[\arccoth{x}]=\frac{1}{1-x^{2}}
\end{equation}
\newpage

\section{Integrals}
\subsection{Techniques of Integration}

U-Substitution
\begin{equation}
\int f(g(x))g'(x) dx = F(g(x))+C
\end{equation}

Integration by Parts
\begin{equation}
\int u\; dv = uv - \int v\; du
\end{equation}

Trig. Integrals
\begin{equation}
\int \sin^{m}{x}\; \cos^{n}{x}\; dx
\end{equation}


\hangindent=3.5em
\hangafter=1
\underline{Case 1:} If power of sin is odd, keep one factor of sin and use the Pythagorean Identity to change $\sin^{2}{x}\rightarrow 1-\cos^{2}{x}$. If power of cos is odd, keep one factor of cos and use the Pythagorean Identity to change $\cos^{2}{x}\rightarrow 1-\sin^{2}{x}$. If both are odd, pick one that will get you to a power of 2 (if possible).\\
\medskip 

\hangindent=3.5em
\hangafter=1
\underline{Case 2:} If both powers are even, use half angle formulas: $\sin^{2}{x}=\frac{1}{2}(1-\cos{2x})$ or $\cos^{2}{x}=\frac{1}{2}(1+\cos{2x})$.

\begin{equation}
\int \tan^{m}{x}\; \sec^{n}{x}\; dx
\end{equation}

\hangindent=3.5em
\hangafter=1
\underline{Case 1:} If power of tan is odd, keep one factor of $\sec{x}\tan{x}$ and use the Pythagorean Identity to change $\tan^{2}{x}\rightarrow \sec^{2}{x}-1$. 
\medskip 

\hangindent=3.5em
\hangafter=1
\underline{Case 2:} If power of sec is even, keep one factor of $\sec^{2}$ and use $\sec^{2}{x} = \tan^{2}{x}+1$.

\begin{equation}
\int \cot^{m}{x}\; \csc^{n}{x}\; dx
\end{equation}

\hangindent=3.5em
\hangafter=1
\underline{Case 1:} If power of cot is odd, keep one factor of $\csc{x}\cot{x}$ and use the Pythagorean Identity to change $\cot^{2}{x}\rightarrow \csc^{2}{x}-1$. 
\medskip 

\hangindent=3.5em
\hangafter=1
\underline{Case 2:} If power of csc is even, keep one factor of $\csc^{2}$ and use $\csc^{2}{x} = \cot^{2}{x}+1$.
\bigskip

Trig. Substitutions
\begin{center}
\begin{tabular}{cccc}

$\sqrt{a^{2}-x^{2}}$ \smallskip & $x=a\sin{\theta}$ \smallskip & $dx=a\cos{\theta}d\theta$ \smallskip & $\sqrt{a^{2}-x^{2}}=a\cos{\theta}$ \smallskip \\
$\sqrt{a^{2}+x^{2}}$ \smallskip & $x=a\tan{\theta}$ \smallskip & $dx=a\sec^{2}{\theta}d\theta$ \smallskip & $\sqrt{a^{2}+x^{2}}=a\sec{\theta}$ \smallskip \\
$\sqrt{x^{2}-a^{2}}$ & $x=a\sec{\theta}$ & $dx=a\sec{\theta}\tan{\theta}d\theta$ & $\sqrt{x^{2}-a^{2}}=a\tan{\theta}$\\

\end{tabular}
\end{center}

Integrating Inverse Functions
\begin{equation}
\int f^{-1}(x) dx = xf^{-1}(x)-(F\circ f^{-1})(x)+C
\end{equation}

Partial Fractions

\newpage

\section{Series}
\subsection{Types of Series}
Geometric Series
\begin{equation}
\sum_{n=1}^{\infty}ar^{n-1}\;  \textrm{or} \; \sum_{n=0}^{\infty}ar^{n}
\end{equation}

\begin{center}
Converges when $-1<r<1$
\end{center}
\begin{equation}
\sum_{n=1}^{\infty}ar^{n-1}= \frac{a}{1-r}
\end{equation}

Harmonic Series
\begin{equation}
\sum_{n=1}^{\infty}\frac{1}{n}=1+\frac{1}{2}+\frac{1}{3}+\frac{1}{4}. . .
\end{equation}
\begin{center}
Diverges
\end{center}

P-Series
\begin{equation}
\sum_{n=1}^{\infty}\frac{1}{n^{p}}=1+\frac{1}{2^{p}}+\frac{1}{3^{p}}+\frac{1}{4^{p}}. . .
\end{equation}
\begin{center}
Converges when $p>1$\\
Diverges when $p\leq1$
\end{center}

Telescoping Series
\begin{equation}
\sum_{n=1}^{\infty}(a_{n}-a_{n-1})=a_{\infty}-a_{0}
\end{equation}

\subsection{Testing for Convergence}
Limit Test
\begin{equation}
\textrm{If} \lim_{n\rightarrow \infty}a_{n}\neq0, \textrm{then}\sum a_{n}\ \textrm{diverges}
\end{equation}

Type of Series
\begin{itemize}
\item Geometric: converges if $\mid r \mid <1$
\item Telescoping: find lim of $a_{\infty}-a_{0}$
\item P-Series: converges when $p>1$
\end{itemize}

Integral Test
\begin{equation}
\textrm{If} \int_{1}^{\infty}a(n)\;dn\ \textrm{converges, then}\sum a_{n} \textrm{converges}
\end{equation}
\begin{equation}
\textrm{If} \int_{1}^{\infty}a(n)\;dn\ \textrm{diverges, then}\sum a_{n} \textrm{diverges}
\end{equation}
\end{flushleft}
\newpage

Comparison Test\\
\hangindent=2cm If all terms of $a_{n}$ are positive and it acts like another series $b_{n}$.
\begin{equation}
\textrm{If}\ a_{n}\leq b_{n}\ \textrm{and} \sum b_{n}\ \textrm{converges, then} \sum a_{n} \textrm{converges}
\end{equation}
\begin{equation}
\textrm{If}\ a_{n}\geq b_{n}\ \textrm{and} \sum b_{n}\ \textrm{diverges, then} \sum a_{n} \textrm{diverges}
\end{equation}
\begin{equation}
\textrm{If}\ \lim_{n\rightarrow\infty} \frac{a_{n}}{b_{n}}\ \textrm{exists, then both series converge or diverge}
\end{equation}

Absolute Convergence\\
\hangindent=2cm For $\sum (-1)^{n}a_{n}$ or $\sum (-1)^{n-1}a_{n}$
\begin{equation}
\textrm{If}\ \sum \vert a_{n} \vert \ \textrm{converges, then} \sum a_{n} \textrm{is absolutely convergent.}
\end{equation}
\begin{equation}
\textrm{If} \lim_{n\rightarrow \infty}a_{n}=0,\ \textrm{then}\sum (-1)^{n}a_{n}\ \textrm{converges}
\end{equation}
\begin{equation}
\textrm{If}\ a_{n+1}\leq a_{n}\ \textrm{or}\ f'(x)<0\ \textrm{then} \sum (-1)^{n}a_{n} \textrm{converges}
\end{equation}

Ratio Test\\
\hangindent=2cm use for factorials or "n"th powers
\begin{equation}
\textrm{If} \lim_{n\rightarrow \infty} \left\vert \frac{a_{n+1}}{a_{n}} \right\vert<1 , \textrm{then}\sum a_{n}\ \textrm{is absolutely convergent.}
\end{equation}
\begin{equation}
\textrm{If} \lim_{n\rightarrow \infty} \left\vert \frac{a_{n+1}}{a_{n}} \right\vert>1 , \textrm{then}\sum a_{n}\ \textrm{is divergent.}
\end{equation}
\begin{equation}
\textrm{If} \lim_{n\rightarrow \infty} \left\vert \frac{a_{n+1}}{a_{n}} \right\vert=1 , \textrm{then}\sum a_{n}\ \textrm{is inconclusive.}
\end{equation}

Root Test\\
\hangindent=2cm use for "n"th powers
\begin{equation}
\textrm{If} \lim_{n\rightarrow \infty} \sqrt[n]{\vert a_{n} \vert} <1 , \textrm{then}\sum a_{n}\ \textrm{is absolutely convergent.}
\end{equation}
\begin{equation}
\textrm{If} \lim_{n\rightarrow \infty} \sqrt[n]{\vert a_{n}\vert } >1 , \textrm{then}\sum a_{n}\ \textrm{is divergent.}
\end{equation}
\begin{equation}
\textrm{If} \lim_{n\rightarrow \infty} \sqrt[n]{\mid a_{n} \mid } =1 , \textrm{then}\sum a_{n}\ \textrm{is inconclusive.}
\end{equation}
\newpage

\subsection{Power Series}
Power Series- a series with some variable "x" raised to some power
\begin{equation}
\sum_{n=0}^{\infty}a_{n}(x-c)^{n}= a_{0}+a_{1}(x-c)+a_{2}(x-c)^{2}
\end{equation}
\begin{center}
P-series centered at c\\
Creates a function of x
\end{center}
\bigskip

Convergence of a Power Series
\begin{itemize}
\item at x=c, Domain: [c,c]
\item for all x, Domain: $(-\infty,\infty)$
\item for some $\mid x-c \mid < R$, Domain (c-R, c+R)
\end{itemize}
\begin{center}
need to check endpoints
\end{center}

\subsection{Taylor Series}
Taylor Series
\begin{equation}
f(x)=\sum_{n=0}^{\infty}\frac{f^{n}(c)}{n!}(x-c)^{n}
\end{equation}
\begin{equation}
=f(c)+f'(c)(x-c)+\frac{f''(c)}{2!}(x-c)^{2}
\end{equation}

\begin{flushleft}
Maclaurin Series- \begin{small} Taylor Series at c=0 \end{small}
\begin{equation}
f(x)=\sum_{n=0}^{\infty}\frac{f^{n}(0)}{n!}(x)^{n}
\end{equation}
\end{flushleft}
\newpage

\subsection{List of Common Taylor Series}


\begin{equation}
\frac{1}{1-x}=1+x+x^{2}+x^{3}+...=\sum_{n=0}^{\infty}x^{n}\ \ \ \ \ \ (1,1)
\end{equation}
\begin{equation}
\frac{1}{(1-x)^{2}}=1+2x+3x^{2}+4x^{3}+...=\sum_{n=0}^{\infty}nx^{n-1}\ \ \ \ \ \ (1,1)
\end{equation}
\begin{equation}
\lnp{1+x}=x-\frac{x^{2}}{2}+\frac{x^{3}}{3}-\frac{x^{4}}{4}...=\sum_{n=0}^{\infty}\frac{(-1)^{n}}{n+1}x^{n+1}\ \ \ \ \ \ (1,1)
\end{equation}
\begin{equation}
\tan^{-1}{x}=x-\frac{x^{3}}{3}+\frac{x^{5}}{5}-\frac{x^{7}}{7}...=\sum_{n=0}^{\infty}\frac{(-1)^{n}}{2n+1}x^{2n+1}\ \ \ \ \ \ (1,1)
\end{equation}
\begin{equation}
\tanh^{-1}{x}=x+\frac{x^{3}}{3}+\frac{x^{5}}{5}+\frac{x^{7}}{7}...=\sum_{n=0}^{\infty}\frac{x^{2n+1}}{2n+1}\ \ \ \ \ \ (1,1)
\end{equation}
\begin{equation}
e^{x}=1+x+\frac{x^{2}}{2!}+\frac{x^{3}}{3!}...=\sum_{n=0}^{\infty}\frac{x^{n}}{n!}\ \ \ \ \ \ (-\infty,\infty)
\end{equation}
\begin{equation}
\sinh{x}=x+\frac{x^{3}}{3!}+\frac{x^{5}}{5!}+\frac{x^{7}}{7!}...=\sum_{n=0}^{\infty}\frac{x^{2n+1}}{2n+1!}\ \ \ \ \ \ (-\infty,\infty)
\end{equation}
\begin{equation}
\cosh{x}=1+\frac{x^{2}}{2!}+\frac{x^{4}}{4!}+\frac{x^{6}}{6!}...=\sum_{n=0}^{\infty}\frac{x^{2n}}{2n!}\ \ \ \ \ \ (-\infty,\infty)
\end{equation}
\begin{equation}
\sin{x}=x-\frac{x^{3}}{3!}+\frac{x^{5}}{5!}-\frac{x^{7}}{7!}...=\sum_{n=0}^{\infty}\frac{(-1)^{n}}{2n+1!}x^{2n+1}\ \ \ \ \ \ (-\infty,\infty)
\end{equation}
\begin{equation}
\cos{x}=1-\frac{x^{2}}{2!}+\frac{x^{4}}{4!}-\frac{x^{6}}{6!}...=\sum_{n=0}^{\infty}\frac{(-1)^{n}}{2n!}x^{2n}\ \ \ \ \ \ (-\infty,\infty)
\end{equation}
\begin{equation}
W(x)=x-x^{2}+\frac{3x^{3}}{2}-\frac{8x^{4}}{3}...=\sum_{n=1}^{\infty}\frac{(-n)^{n-1}}{n!}x^{n}\ \ \ \ \ \ \left(-\frac{1}{e},\frac{1}{e}\right)
\end{equation}


\newpage

\section{Vectors}
\subsection{Basic Properties}
\begin{flushleft}
A vector $\vec{v}$ with an initial point at the origin and an terminal point P $(v_{1},v_{2})$ is called an position vector and is denoted $\vecp{v_{1},v_{2}}$.\\
\bigskip

Magnitude
\begin{equation}
\magp{\vec{v}}=\sqrt{v_{1}^{2}+v_{2}^{2} + \dots}
\end{equation}

Basic Properties
\begin{equation}
\vec{a}+\vec{b}= \vecp{a_{1}+b_{1}, a_{2}+b_{2}, \dots}
\end{equation}
\begin{equation}
C \cdot \vec{v} = \vecp{C \cdot v_{1}, C \cdot v_{2, \dots}}
\end{equation}

Unit Vector
\begin{equation}
\hat{u}=\frac{\vec{v}}{\magp{\vec{v}}}
\end{equation}

Standard Basis Vectors
\begin{equation}
\hat{i}=\vecp{1, 0, 0}
\end{equation}
\begin{equation}
\hat{j}=\vecp{0, 1, 0}
\end{equation}
\begin{equation}
\hat{k}=\vecp{0, 0, 1}
\end{equation}


\subsection{Dot Product}
Adds the products of corresponding components of two vectors. Gives a scalar.
\begin{equation}
\vec{a} \cdot \vec{b}= a_{1}b_{1}+a_{2}b_{2}+a_{3}b_{3}+\dots= C
\end{equation} 

Properties
\begin{equation}
\vec{v} \cdot \vec{w} = \vec{w} \cdot \vec{v} 
\end{equation}
\begin{equation}
\vec{v} \cdot (\vec{u} + \vec{w}) = \vec{v} \cdot \vec{u} +\vec{v} \cdot \vec{w}
\end{equation}
\begin{equation}
(C\vec{v}) \cdot \vec{w} = C(\vec{w} \cdot \vec{v})=  \vec{v} \cdot (C\vec{w})
\end{equation}
\begin{equation}
\vec{0} \cdot \vec{v}= 0
\end{equation}
\begin{equation}
\vec{v} \cdot \vec{v} = \magp{\vec{v}}^{2}
\end{equation}


Another Definition ($\theta$ is the angle between the vectors)
\begin{equation}
\vec{v} \cdot \vec{w} = \magp{\vec{v}} \magp{\vec{w}} \cos{\theta}
\end{equation}
\begin{equation}
\cos{\theta }= \frac{\vec{v} \cdot \vec{w}}{\magp{\vec{v}} \magp{\vec{w}}}
\end{equation}
\begin{equation}
\theta= \arccos{\frac{\vec{v} \cdot \vec{w}}{\magp{\vec{v}} \magp{\vec{w}}}}
\end{equation}
If $\theta=0$ or $\pi$, the vectors are parallel.\\ 
If $\theta=\frac{\pi}{2}$, the vectors are orthogonal.



\newpage

\subsection{Cross Product}
\begin{equation}
\vec{a} \times \vec{b}=
\begin{bmatrix}
\hat{i} & \hat{j} & \hat{k} \\
a_{1} & a_{2} & a_{3} \\
b_{1} & b_{2} & b_{3} \\
\end{bmatrix} 
=\begin{bmatrix}
a_{2} & a_{3} \\
b_{2} & b_{3} \\
\end{bmatrix} \hat{i} -
\begin{bmatrix}
a_{1} & a_{3} \\
b_{1} & b_{3} \\
\end{bmatrix}\hat{j}+
\begin{bmatrix}
a_{1} & a_{2} \\
b_{1} & b_{2} \\
\end{bmatrix} \hat{k}
\end{equation}

Properties\\
\bigskip
1. $\vec{a} \times \vec{b}$ is a vector.\\
2. $\vec{a} \times \vec{b}$ is orthogonal to both $\vec{a}$ and $\vec{b}$.\\
3. $\vec{b} \times \vec{a}$ is orthogonal to both $\vec{a}$ and $\vec{b}$ but other direction.\\
4. $\vec{a} \times \vec{b} = -(\vec{b} \times \vec{a})$\\

Another Definition
\begin{equation}
\magp{\vec{a} \times \vec{b}}= \magp{\vec{a}} \magp{\vec{b}} \sin{\theta}
\end{equation}

Area of a Triangle with Sides $\vec{u}$ and $\vec{v}$.
\begin{equation}
A_{t}=\frac{1}{2} \magp{\vec{u} \times \vec{v}}
\end{equation}

Area of a Parallelogram with Sides $\vec{u}$ and $\vec{v}$.
\begin{equation}
A_{p}=\magp{\vec{u} \times \vec{v}}
\end{equation}

Volume of a Parallelepypite with Sides $\vec{a}$, $\vec{b}$, and $\vec{c}$.
\begin{equation}
V_{p}=\absp{\vec{a} \cdot (\vec{b} \times \vec{c})}
\end{equation}
\newpage

\section{Vector Functions}
\subsection{Basics and Properties}
Sketches a line through space with vectors starting at the origin.
\begin{equation}
\vec{r}(t)= f(t)\hat{i}+g(t)\hat{j}+h(t)\hat{k}
\end{equation}
Properties
\begin{equation}
\lim_{t \rightarrow a}\vec{r}(t)= \lim_{t \rightarrow a}f(t)\hat{i}+\lim_{t \rightarrow a}g(t)\hat{j}+\lim_{t \rightarrow a}h(t)\hat{k}
\end{equation}
\begin{equation}
\frac{d}{dt}\vec{r}(t)= f'(t)\hat{i}+g'(t)\hat{j}+h'(t)\hat{k}
\end{equation}
\begin{equation}
\int\vec{r}(t)= F(t)\hat{i}+G(t)\hat{j}+H(t)\hat{k}+C
\end{equation}

Projectile Motion
\begin{equation}
\vec{r}(t)= (\vec{v}_{0}\cos{\alpha})t\hat{i}+[h+(\vec{v}_{0}\sin{\alpha})t-\frac{1}{2}gt^{2}]\hat{j}
\end{equation}
$$ \vec{v}_{0}= \textrm{initial velocity}$$
$$ \alpha= \textrm{angle of inclination}$$
$$ h = \textrm{height above plane}$$
$$ g = \textrm{acceleration of gravity}$$




\subsection{Arc Length}
\begin{equation}
L= \int_{a}^{b}\sqrt{[f'(t)]^{2}+[g'(t)]^{2}+[h'(t)]^{2}} dt
\end{equation}
\begin{equation}
L= \int_{a}^{b}\magp{\vec{r}'(t)} dt
\end{equation}

Reparameterizing by Arc Length
\end{flushleft}
\begin{equation}
\textrm{Find} \ \  S(t)=\int \magp{\vec{r}'(t)} dt
\end{equation}
\begin{equation}
\textrm{Solve for t in terms of S} \ \ t=u(S)
\end{equation}
\begin{center}
Substitute back into vector function
\end{center}
\begin{equation}
\vec{r}(t)= f(u(S))\hat{i}+g(u(S))\hat{j}+h(t)\hat{u(S)}
\end{equation}
\newpage

\begin{flushleft}
\subsection{TNB Frames}
Frenet-Serret Frames\\
\bigskip
$\vec{T}$angent: Direction particle is heading.\\
$\vec{N}$ormal: Direction particle is turning.\\
$\vec{B}$inormal: Direction particle is twisting.\\
\bigskip

How to find TNB
\begin{equation}
\vec{T}(t)=\frac{\vec{r}'(t)}{\magp{\vec{r}'(t)}}
\end{equation}
\begin{equation}
\vec{N}(t)=\frac{\vec{T}'(t)}{\magp{\vec{T}'(t)}}
\end{equation}
\begin{equation}
\vec{B}(t)=\vec{T}(t) \times \vec{N}(t)
\end{equation}

\subsection{Curvature, Torsion, and Osculating Circles}
Curvature: "failure to be a line"
\begin{equation}
\kappa = \frac{\textrm{change in tangent vector}}{\textrm{arc length}}= \magp{\vec{T}'(S)}
\end{equation}
\begin{equation}
\kappa = \frac{\magp{\vec{T}'(t)}}{\magp{\vec{r}'(t)}}
\end{equation}
\begin{equation}
\kappa = \frac{\magp{\vec{r}'(t) \times \vec{r}''(t)}}{\magp{\vec{r}'(t)}^{3}} \ \ \textrm{"only for polynomials"}
\end{equation}

Torsion: "failure to be contained in a plane"

\begin{equation}
\tau = \frac{(\vec{r}' \times \vec{r}'') \cdot \vec{r}'''}{\magp{\vec{r}' \times \vec{r}''}^{2}}
\end{equation}
\begin{equation}
\tau =\frac{-d\vec{B}}{ds} \cdot \vec{N} 
\end{equation}

At one point on a curve, there will be a circle that fits the curve "best" called a osculating circle with a radius of curvature \begin{equation}\rho= \frac{1}{\kappa} \end{equation} The plane that contains $\vec{N}$ and $\vec{B}$ at the point is called the normal plane and contains all vectors orthogonal to $\vec{T}$.
\newpage

\section{Multivariable Functions}
\subsection{Lines and Planes}
Lines need a point $(x_{0},\ y_{0},\ z_{0})$ and direction vector $\vecp{a,\ b,\ c}$.\\
\bigskip

Parametric Equation for a Line
\begin{equation}
x=at+x_{0},\ \ \ y=bt+y_{0},\ \ \ z=ct+z_{0}
\end{equation}

Symmetric Equation of a Line
\begin{equation}
\frac{x-x_{0}}{a}=\frac{y-y_{0}}{b}=\frac{z-z_{0}}{c}
\end{equation}

\bigskip
\bigskip


Planes need a point $P_{0}(x_{0},\ y_{0},\ z_{0})$ and normal vector $\vec{n}=\vecp{a,\ b,\ c}$.\\
\bigskip

Standard Form of a Plane
\begin{equation}
a(x-x_{0})+b(y-y_{0})+c(z-z_{0})=0
\end{equation}
General Form
\begin{equation}
ax+by+cz=d
\end{equation}

Angle Between Line and Plane
\begin{equation}
90^{\circ}-\arccos{\frac{\absp{\vec{n} \cdot \vec{v}}}{\magp{\vec{n}} \magp{\vec{v}}}}
\end{equation}

\subsection{Distances}
Point $P_{T}(x_{T},\ y_{T},\ z_{T})$ and a Plane $P_{0}(x_{0},\ y_{0},\ z_{0})\ \ \vec{n}=\vecp{a,\ b,\ c}$.
\begin{equation}
D=\left\vert \frac{ax_{T}+by_{T}+cz_{T}+d}{\sqrt{a^{2}+b^{2}+c^{2}}}  \right\vert
\end{equation}

Parallel Planes
\begin{equation}
D= \frac{\absp{d_{1}-d_{2}}}{\sqrt{a^{2}+b^{2}+c^{2}}}  
\end{equation}

Point P and Line Q and $\vec{v}$
\begin{equation}
D=\frac{\magp{\vec{PQ} \times \vec{v}}}{\magp{\vec{v}}}
\end{equation}

Between Skewed Lines\\
\bigskip
Find direction vectors $\vec{v}_{1}$ and $\vec{v}_{2}$. Find common normal $\vec{n}=\vec{v}_{1} \times \vec{v}_{2}$. Pick a point on each line $P_{1}$ and $P_{2}$. Take $P_{1}$ and $\vec{n}$ and $P_{2}$ and $\vec{n}$ to make equations of planes and solve for the distance between parallel planes. 
\newpage


\subsection{Cylinder and Surfaces}
Cylinder
\begin{enumerate}
\item Equations only have two variables.
\item Directed along axis of missing variable.
\item Does not change along direction axis.
\end{enumerate}


Surfaces
\begin{enumerate}
\item Has three variables.
\item Traces occur on coordinate planes or parallel to.
\item Shape changes along the axis.
\end{enumerate}
\bigskip

Ellipsoid
\begin{equation}
\frac{x^2}{a^2}+\frac{y^2}{b^2}+\frac{z^2}{c^2}=1
\end{equation}

One-Sheet Hyperbolas
\begin{equation}
\frac{x^2}{a^2}+\frac{y^2}{b^2}-\frac{z^2}{c^2}=1
\end{equation}

Two Sheet Hyperbola 
\begin{equation}
\frac{x^2}{a^2}-\frac{y^2}{b^2}-\frac{z^2}{c^2}=1
\end{equation}

Cones
\begin{equation}
\frac{x^2}{a^2}+\frac{y^2}{b^2}-\frac{z^2}{c^2}=0
\end{equation}

Paraboloids
\begin{equation}
\frac{x^2}{a^2}+\frac{y^2}{b^2}=cz
\end{equation}

Hyperbolic Paraboloid 
\begin{equation}
\frac{x^2}{a^2}-\frac{y^2}{b^2}=cz
\end{equation}

\subsection{Limits of Multivariable Functions}
\begin{equation}
\lim_{(x,y)\rightarrow (a,b)}f(x,y)=L
\end{equation}
Problem: infinite ways to approach (a,b). \\
To prove that L doesn't exist, its sufficient to show different values of L from different lines through (a,b).
\newpage

\section{Partial Derivatives}
\subsection{Basics}
\begin{equation}
\parx{f}= \parx{z}=f_{x}=z_{x}
\end{equation}
Holds any other variables constant.\\
Slope of tangent line in x-direction.

\begin{equation}
\pary{f}=\pary{z}=f_{y}=z_{y}
\end{equation}
Holds any other variables constant.\\
Slope of tangent line in y-direction.\\
\bigskip

Higher Derivatives
\begin{equation}
\frac{\partial^2 f}{\partial x^2},\ \ \ \frac{\partial^2 f}{\partial y^2},\ \ \ \frac{\partial^2 f}{\partial x \partial y}=\frac{\partial^2 f}{\partial y \partial x}
\end{equation}

h(x,y) is called a Harmonic Function if it satisfies the Laplace Equation
\begin{equation}
h_{xx}+h_{yy}=0
\end{equation}

Differential
\begin{equation}
dz=f_{x}dx+f_{y}dy
\end{equation}

Directional Derivative $\hat{u}=\vecp{u_{1},\ u_{2}}$
\begin{equation}
D_{\hat{u}}f(x,y)=f_{x}u_{1}+f_{y}u_{2}
\end{equation}

\subsection{Multivariable Chain Rule}
Suppose that "f" is a function of "x" and "y" where "x" and "y" are functions of some variable "t."
\begin{equation}
\frac{df}{dt}=\parx{f}\frac{dx}{dt} + \pary{f}\frac{dy}{dt}
\end{equation}


"w" is a function of "x," "y," and "z." "x," "y," and "z" are based of variables "u" and "v."
\begin{equation}
\frac{\partial w}{\partial u}= \parx{w}\frac{\partial x}{\partial u}+\pary{w}\frac{\partial y}{\partial u}+\frac{\partial w}{\partial z}\frac{\partial z}{\partial u}
\end{equation}


\subsection{Gradient}
\begin{equation}
\bigtriangledown f(x,y)= f_{x}\hat{i}+f_{y}\hat{j}
\end{equation}
\begin{equation}
D_{\hat{u}}f(x,y)= \bigtriangledown f(x,y) \cdot \hat{u}
\end{equation}
\newpage

Properties of the $\bigtriangledown$
\begin{enumerate}
\item If $\bigtriangledown f=\vec{0}$ then $D_{\hat{u}}f=0$ of any $\hat{u}$.
\item $D_{\hat{u}}f(x,y)$ has its max value of $\magp{\bigtriangledown f(x,y)}$ only when $\hat{u}=C \cdot \bigtriangledown f$.
\item $D_{\hat{u}}f(x,y)$ has its minimum value of $-\magp{\bigtriangledown f(x,y)}$ at $\theta = \pi$ from max value.
\end{enumerate}

\subsection{Local Extrema of Functions}
Second Derivative Test
\begin{equation}
D(x,y)=f_{xx}f_{yy}-(f_{xy})^{2}
\end{equation}
\begin{enumerate}
\item $D(a,b) > 0$\ \ $f_{xx}(a,b)>0$\ \ Concave up; relative min
\item $D(a,b) > 0$\ \ $f_{xx}(a,b)<0$\ \ Concave down; relative max
\item $D(a,b) < 0$ saddle/inflection point
\item $D(a,b) = 0$ no conclusions can be drawn
\end{enumerate}

\subsection{LaGrange Multipliers}
Idea: to find the local min/max of a surface $f(x,y)=z$ along a constraint $g(x,y)=c$.

\begin{equation}
\bigtriangledown f(x,y)= \lambda \bigtriangledown g(x,y)
\end{equation}
Plug $\lambda$ back into equations used to find lambda to a coordinate.\\
Plug back into f(x,y) and do second derivative test to find out if its a max or min.




\newpage
\section{Multiple Integrals}
\subsection{Double Integrals}
Definition:
\begin{equation}
\lim_{m,n \rightarrow \infty }\sum_{i=1}^m \sum_{j=1}^n\ f(x_{ij}, y_{ij})\ \Delta x \Delta y\ = \iint \limits_R  f(x,y)\ dxdy 
\end{equation}

Fubini's Theorem for General Regions
\begin{equation}
\int \limits_a^b \int \limits_{g_1(x)}^{g_2(x)} f(x,y)\ dydx = \int \limits_c^d \int \limits_{h_1(x)}^{h_2(x)} f(x,y)\ dxdy
\end{equation}

Double Integrals in Polar Coordinates
\begin{equation}
\iint \limits_R  f(x,y)\ dA = \int \limits_\alpha^\beta \int \limits_{g_1(\theta)}^{g_2(\theta)} f(r\cos \theta, r\sin \theta) \cdot r\ drd\theta 
\end{equation}

\subsection{Triple Integrals}
\begin{equation}
\iiint \limits_T f(x,y,z)\ dV
\end{equation}
\begin{enumerate}
\item For $\displaystyle \iiint \limits_T $ define T between two surfaces. This takes care of the first $\displaystyle \int$ and leaves $\iint \limits_R$ just like before. But, R can be on XY, YZ, or XZ planes.\\
\indb \underline{ Two Tasks and Two Regions}
\begin{enumerate}
 \item A area in $\mathbb{R}-3$
 \item Then a very specific $\mathbb{R}-2$ region on the coordinate plane.
\end{enumerate}
\item $\displaystyle \iiint \limits_T f(x,y,z) dV$ can be evaluated in any order.
\item x/y/z Simple\\
\begin{itemize}
\item \underline{z-simple}: Define $\mathbb{R}-3$ region between two $z=f(x,y)$ functions, R will be in the xy-plane. 
\item \underline{y-simple}: Define $\mathbb{R}-3$ region between two $z=f(x,z)$ functions, R will be in the xz-plane. 
\item \underline{x-simple}: Define $\mathbb{R}-3$ region between two $z=f(y,z)$ functions, R will be in the yz-plane. 
\end{itemize}
\end{enumerate}
\newpage
\subsection{C.O.M and Moments of Inertia}
If an object has a density of $\rho (x,y,z) $ at any point $(x,y,z)$ then the mass of the plate is:
\begin{equation}
m=\iiint \limits_T \rho (x,y,z)\ dV
\end{equation}

\noindent \underline{First Moments of Mass}:

\begin{equation}
M_{yz} = \iiint \limits_T  x\rho (x,y,z)\ dV, \ \ \ \ \ \bar{x}=\frac{M_{yz}}{m}
\end{equation}
\begin{equation}
M_{xz} = \iiint \limits_T  y\rho (x,y,z)\ dV, \ \ \ \ \ \bar{y}=\frac{M_{xz}}{m}
\end{equation}
\begin{equation}
M_{xy} = \iiint \limits_T  z\rho (x,y,z)\ dV, \ \ \ \ \ \bar{z}=\frac{M_{xy}}{m}
\end{equation}

\begin{equation}
\textrm{C.O.M}\ \ (\bar{x},\ \bar{y},\ \bar{z})
\end{equation}

\noindent \underline{Second Moments of Inertia}:

\begin{equation}
I_x = \iiint \limits_T  (y^2+z^2)\rho (x,y,z)\ dV
\end{equation}
\begin{equation}
I_y = \iiint \limits_T  (x^2+z^2)\rho (x,y,z)\ dV
\end{equation}
\begin{equation}
I_z = \iiint \limits_T  (x^2+y^2)\rho (x,y,z)\ dV
\end{equation}

\noindent \underline{Radius of Gyration}:
\begin{equation}
k_x = \sqrt{\frac{I_x}{m}}
\end{equation}
\begin{equation}
k_y = \sqrt{\frac{I_y}{m}}
\end{equation}
\begin{equation}
k_z = \sqrt{\frac{I_z}{m}}
\end{equation}


\newpage
\subsection{Cylindrical and Spherical Coordinates}


\inda \underline{Cylindrical:} $\ \ x=r\cos\theta \ \ y=r\sin\theta \ \ x^2+y^2=r^2 \ \ z=z $
\begin{equation}
\iiint \limits_T  f(x,y,z)\ dV = \int \limits_{\theta_1}^{\theta_2} \int \limits_{r_1}^{r_2} \int \limits_{z_1}^{z_2} f(r\cos\theta,r\sin\theta, z)\ dz rdrd\theta
\end{equation}

\inda \underline{Spherical:}
$$ x=\rho\sin\phi\cos\theta$$
$$ y=\rho\sin\phi\sin\theta$$
$$ z=\rho\cos\phi$$
$$ r=\rho\sin\phi$$
$$ x^2+y^2+z^2=\rho^2$$
\begin{equation}
\iiint \limits_T  f(x,y,z)\ dV = \int \limits_{\theta_1}^{\theta_2} \int \limits_{\phi_1}^{\phi_2} \int \limits_{\rho1}^{\rho_2} f(\rho\sin\phi\cos\theta,\rho\sin\phi\sin\theta, \rho\cos\phi)\ \rho^2\sin\phi\ d\rho d\phi d\theta
\end{equation}


\subsection{The Jacobian}
For
\begin{equation}
\iint \limits_R f(x,y)\ dA \Rightarrow \iint f(g(u,v),h(u,v)) \cdot J \cdot dudv
\end{equation}
J the Jacobian:
\setlength{\delimitershortfall}{0pt}
\begin{equation}
J=\frac{\partial (x,y)}{\partial(u,v)}=\begin{vmatrix}
\frac{\partial x}{\partial u} & \frac{\partial x}{\partial v} \\[2ex]
\frac{\partial y}{\partial u} & \frac{\partial y}{\partial v} 
\end{vmatrix}
\end{equation}
Generalizes to
\begin{equation}
\begin{vmatrix}
\frac{\partial x}{\partial u} & \frac{\partial x}{\partial v} & \frac{\partial x}{\partial w} & \hdots \\[2ex]
\frac{\partial y}{\partial u} & \frac{\partial y}{\partial v} & \frac{\partial y}{\partial w} & \hdots \\[2ex]
\frac{\partial z}{\partial u} & \frac{\partial z}{\partial v} & \frac{\partial z}{\partial w} & \hdots \\[2ex]
\vdots & \vdots &\vdots & \ddots
\end{vmatrix}
\end{equation}

\newpage
\section{Vector Calculus}
\subsection{Vector Fields}
\underline{Vector Field}: an object which gives a vector for each coordinate point
\begin{equation}
F(x,y,z)= P\hat{i} + Q\hat{j} + R\hat{k}
\end{equation}
\indn For which $P,$ $Q,$ and $R$ are defined functions.
\bigskip


 \underline{Conservative Vector Field}:\\
\indd $F(x,y,z)$ is conservative V.F. if $F(x,y,z) = \bigtriangledown f(x,y,z)$\\
\indf For some $f(x,y,z)$ called a potential function.

\subsection{Divergence and Curl}
Divergence and curl are two characteristics of how flow is behaving on a vector field in a small neighborhood aroud a given point ``$P$."

\bigskip
\noindent \underline{Divergence}- a measurement of how much fluid enters the neighborhood around ``$P$" compared to how much leaves. \\

\begin{itemize}
\item If more fluid enters region than leaves, then the divergence will be negative. Think of this as fluid gathering at a  point. 
\item If the same amount of fluid/flow enters as leave, divergence is ``0'' at ``P'' called incompressible.
\item If more fluid/flow leaves than enters, divergence is positive at ``P.'' Think of fluid leaving the point called divergent.
\end{itemize}

For $F(x,y,z) = P\hat{i} + Q\hat{j} + R\hat{k}, \ \ \ \ \bigtriangledown = \parx{}\hat{i} + \pary{}\hat{j} + \parz{}\hat{k}$
\begin{equation}
\vdiv F = \bigtriangledown \cdot F = \parx{P} + \pary{Q} + \parz{R}
\end{equation}

\noindent \underline{Curl}- measure of rotation of the vector field in the neighborhood around ``P''

\begin{itemize}
\item If curl is positive at a point, the fluid is spinning counter-clockwise.
\item If curl is negative at a point, the fluid is spinning clockwise.
\item If curl is 0, there is no rotation a ``P,'' called irrotational.
\end{itemize}

\begin{equation}
\vcurl F = \bigtriangledown \times F = \begin{vmatrix}
\hat{i} & \hat{j} & \hat{k} \\
\parx{} & \pary{} & \parz{} \\
P & Q & R
\end{vmatrix}
\end{equation}
\begin{equation}
= \left( \pary{R} - \parz{Q} \right)\hat{i} - \left( \parx{R} - \parz{P} \right)\hat{j} + \left( \parx{Q} - \pary{P} \right)\hat{k}
\end{equation}

\newpage
\subsection{Line Integrals}
Integrating over any curve instead of just the x axes.

\begin{equation}
m = \int \limits_c f(x,y)\ ds
\end{equation}
To define ``c'' use parametric equations\\
``ds'' means with respect to arc length, little length of the curve\\
$f(x,y)$ height above each point on ``c''\\
$f(x,y) \cdot ds $ = Area

$$
m = \int \limits_c 2\ \textrm{var.} \rightarrow \int \limits_c 1\ \textrm{var.}
$$
$$
f(x,y) = f(x(t),y(t))
$$
$$
ds = \sqrt{[x'(t)]^2 + [y'(t)]^2}\ dt
$$
$$
c = \vec{r}(t) = x(t)\hat{i} + y(t)\hat{j}
$$
$$
\vec{r}'(t) = x'(t)\hat{i} + y'(t)\hat{j}
$$
$$
\magp{\vec{r}'(t)} = \sqrt{[x'(t)]^2 + [y'(t)]^2}
$$
\begin{equation}
ds = \magp{\vec{r}'(t)}\ dt
\end{equation}
\begin{equation}
\int \limits_c f(x,y)\ ds = \int \limits_{t=a}^{t=b} f(x(t),y(t)) \ \magp{\vec{r}'(t)}\ dt
\end{equation}

\underline{Other possibilities}
\begin{enumerate} 
\item Instead of $\int ds$ you get \begin{equation}
\int dx + dy + dz
\end{equation}
Substitute $\frac{dx}{dt}$ from c into the integral as $dx = x'(t)\ dt $.
\item If ``c'' isn't smooth split it into sections \begin{equation}
m = \int \limits_{c_1} + \int \limits_{c_2}
\end{equation}
\end{enumerate}

\underline{Line Integral through a Vector Field}:\\
\indb For $F(x,y,z) = F(x(t),y(t),z(t))$ \& $\vec{r}(t) = x(t)\hat{i} + y(t)\hat{j} + z(t)\hat{k}, \ a\leq t \leq b $
\begin{equation}
\int \limits_a^b F(\vec{r}(t)) \cdot \vec{r}'(t)\ dt
\end{equation}
\begin{equation}
w = \int \limits_c F \cdot dr
\end{equation}

\newpage
\subsection{Line Integral on a Conservative Vector Field}
``F'' is conservative if F = $\bigtriangledown f\ $ for some function.

\bigskip

\underline{Conservative Vector Field}: no matter how you set from point a to point b, it is the same amount of work. The line integral is independent of path. We don't need to define a curve.

\bigskip
\underline{Fundamental Theorem of Line Integrals}
\begin{equation}
\int \limits_c F \cdot dr = \int \limits_c \bigtriangledown f \cdot dr = f(b) - f(a)
\end{equation}

\inda \underline{\#1}: Show ``F'' is conservative

\bigskip 
\indd $F(x,y) = P\hat{i} + Q\hat{j}$\\
\indd If $\pary{P} = \parx{Q}$ then F is conservative.
\bigskip

\indd $(x,y,z) P\hat{i} + Q\hat{j} + R\hat{k}$\\
\indd Prove $\vcurl F = 0$
\bigskip

\inda \underline{\#2}: $\bigtriangledown f = \parx{f}\hat{i} + \pary{f}\hat{j} + \parz{f}\hat{k}$

\bigskip
\indd Integrate a term\\
\indd Differentiate with respect to another variable \\
\indd Solve for a the extra function.


\subsection{Green's Theorem}
What if $ \int Pdx + Qdy $ has a curve ``C'' that encloses a region on a plane and ``C'' is a simple, closed, curve that is traveled in the counter-clockwise, then the $\int $ becomes $\oint$

\bigskip
\underline{Green's Theorem}:
\begin{equation}
\oint \limits_c Pdx + Qdy = \iint \limits_R \parx{Q} - \pary{P}\ dA
\end{equation}

\bigskip
\underline{Area Formula}:
\begin{equation}
A = \frac{1}{2} \oint \limits_c -y dx + x dy
\end{equation}

\newpage
\subsection{Surface Integrals}
\begin{equation}
m= \iint \limits_S f(x,y,z)\ dS
\end{equation}

Defining one variable as the surface:
\begin{equation}
\iint \limits_S f(x,y,z)\ dS = \iint \limits_R f(x,y,g(x,y))\ \sqrt{g_x^2 + g_y^2 + 1}\ dA
\end{equation}
\begin{equation}
\iint \limits_S f(x,y,z)\ dS = \iint \limits_R f(x,g(x,z),z)\ \sqrt{g_x^2 + g_z^2 + 1}\ dA
\end{equation}
\begin{equation}
\iint \limits_S f(x,y,z)\ dS = \iint \limits_R f(g(y,z),y,z)\ \sqrt{g_y^2 + g_z^2 + 1}\ dA
\end{equation}

\underline{Parametric Surfaces}:\\
\indb For f(x,y,z) \&
$$ x = x(u,v) $$
$$ y = y(u,v) $$
$$ z = z(u,v) $$
\begin{equation}
f(x,y,z) = \vec{r}(u,v) = x(u,v)\hat{i} + y(u,v)\hat{j} + z(u,v)\hat{k}
\end{equation}
\underline{Surface Area}:
\begin{equation}
\iint \limits_D \magp{\vec{r}_u \times \vec{r}_v }\ dA
\end{equation}
\indb ``D'' is the domain of the parameters of ``u'' and ``v''

\bigskip
\underline{Parametric Surface Integral}:
\begin{equation}
\iint \limits_D f(\vec{r}(u,v))\ \magp{\vec{r}_u \times \vec{r}_v }\ dA
\end{equation}
\begin{equation}
\iint \limits_S P\ dydx + Q\ dzdy + R\ dxdy = \iint \limits_D \left( P\frac{\partial (y,z)}{\partial(u,v)}+Q\frac{\partial (z,x)}{\partial(u,v)}+R\frac{\partial (x,y)}{\partial(u,v)} \right)\ du\ dv
\end{equation}

\underline{Flux}: If ``F,'' a vector field contains a surface ``S,'' then F describes the velocity of the flow/field at any point across a surface. The rate of flow across the surface is called flux. 
\begin{equation}
F \cdot \vec{n}\ A(S)
\end{equation} For a small area of S. Across an entire surface it becomes:
\begin{equation}
\iint \limits_S F \cdot d\textbf{S} = \iint \limits_S F \cdot \vec{n} dS 
\end{equation}
\begin{equation}
d\textbf{S} = \vec{n} dS
\end{equation}
\newpage
For $z=g(x,y)$ and a vector field $F = P\hat{i} + Q\hat{j} + R\hat{k} $
\begin{equation}
\iint \limits_S F \cdot dS = \iint \limits_D  \left( -Pg_x - Qg_y + R\right)dA
\end{equation}
\begin{equation}
\iint \limits_S F \cdot dS = \iint \limits_D F(\vec{r}(u,v)) \cdot (\vec{r}_u \times \vec{r}_v \ dA
\end{equation}
\underline{Divergence Theorem}: For when ``S'' is a simple closed surface
\begin{equation}
\textrm{Flux} = \iint \limits_S F \cdot dS = \iiint \limits_T \vdiv F\ dV
\end{equation}

\subsection{Stoke's Theorem}
If a curve ``c'' is not contained in a plane. (Green's Theorem except the region becomes a surface in 3-D)
\begin{equation}
\oint \limits_c F \cdot dr = \iint \limits_S \vcurl F \cdot \vec{n}\ dS
\end{equation}
When evaluating, thinks of $\vcurl F$ as a vector field and use surface integrals.
\begin{equation}
\iint \limits_S \vcurl F \cdot \vec{n}\ dS = \iiint \limits_T \vdiv \left( \vcurl F \right)\ dV
\end{equation}

\end{flushleft}
\end{document}
